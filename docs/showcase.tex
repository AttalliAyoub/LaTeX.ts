\documentclass{article}

\usepackage{comment, multicol}
\usepackage{hyperref}

\usepackage{calc,pict2e,picture}
\usepackage{textgreek,textcomp,gensymb,stix}

\setcounter{secnumdepth}{2}

\title{\LaTeX.js Showcase}
\author{made with $\varheartsuit$ by Michael Brade}
\date{2017--2020}


\begin{document}

\maketitle


\begin{abstract}
This document will show most of the features of \LaTeX.js while at the same time being a gentle introduction to \LaTeX.
In the appendix, the API as well as the format of custom macro definitions in JavaScript will be explained.
\end{abstract}


\section{Characters}

It is possible to input any UTF-8 character either directly or by character code
using one of the following:

\begin{itemize}
    \item \texttt{\textbackslash symbol\{"00A9\}}: \symbol{"00A9}
    \item \verb|\char"A9|: \char"A9
    \item \verb|^^A9 or ^^^^00A9|: ^^A9 or ^^^^00A9
\end{itemize}

\bigskip

\noindent
Special characters, like those:
\begin{center}
\$ \& \% \# \_ \{ \} \~{} \^{} \textbackslash % \< \>  \"   % TODO cannot be typeset
\end{center}
%
have to be escaped.

More than 200 symbols are accessible through macros. For instance: 30\,\textcelsius{} is
86\,\textdegree{}F. See also Section~\ref{sec:symbols}.



\section{Spaces and Comments}

Spaces and comments, of course, work just as they do in \LaTeX.
This is an            % stupid
% Better: instructive <----
example: Supercal%
                ifragilist%
    icexpialidocious

It does not matter whether you enter one or several     spaces after a word, it
will always be typeset as one space---unless you force several spaces, like\ \ now.

New \TeX users may miss whitespaces after a command. % renders wrong
Experienced \TeX{} users are \TeX perts, and know how to use whitespaces. % renders correct

Longer comments can be embedded in the \texttt{comment} environment:
This is another  \begin  {comment}
rather stupid,
but helpful
\end
{comment}
example for embedding comments in your document.



\section{Dashes and Hyphens}

\LaTeX\ knows four kinds of dashes. Access three of them with different numbers
of consecutive dashes. The fourth sign is actually not a dash at all---it is the
mathematical minus sign:

\begin{quote}
    daughter-in-law, X-rated\\
    pages 13--67\\
    yes---or no? \\
    $0$, $1$ and $-1$
\end{quote}
%
The names for these dashes are: ‘-’ hyphen, ‘--’ en-dash, ‘---’ em-dash,
and ‘$-$’ minus sign. \LaTeX.js outputs the actual true unicode character for those
instead of using the hypen-minus.



\section{Text and Paragraphs, Ligatures}

An empty line starts a new paragraph, and so does \verb|\par|.
\par Like this. A new line usually starts automatically when the previous one is
full. However, using \verb+\newline+ or \verb|\\|,\newline one can force \\ to start a new line.

Ligatures are supported as well, for instance:

\begin{center}
fi, fl, ff, ffi, ffl \dots{} instead of f\/i, f\/l, f\/f\/l \dots
\end{center}

Use \texttt{\textbackslash\slash} to prevent a ligature.



\begin{multicols}{2}[\subsection{Multicolumns}]

The multi-column layout, using the \texttt{multicols} environment, allows easy
definition of multiple columns of text---just like in newspapers. The first
and mandatoriy argument specifies the number of columns the text should be divided into.

It is often convenient to spread some text over all columns, just before the multicolumn
output. In \LaTeX, this was needed to prevent any page break in between. To achieve this,
the \texttt{multicols} environment has an optional second argument which can be used for
this purpose.

For instance, this text you are reading now was started with the argument
\texttt{\textbackslash subsection\{Multicolumns\}}.

\end{multicols}



\section{Boxes}

\LaTeX.js supports most of the standard \LaTeX\ boxes.

\medbreak

\noindent\fbox{\verb|\mbox{|\emph{text}\verb|}|}

\smallbreak

We already know one of them: it's called \verb|\mbox|. It simply packs up a series of boxes into another one, and
can be used to prevent \LaTeX.js from breaking two words. As boxes can be put inside boxes, these horizontal box
packers give you ultimate flexibility.

\bigbreak

% yes, unlike in LaTeX, you may put \verb|| anywhere, here it is inside an \fbox :)
\noindent\fbox{
\verb|\makebox[|\emph{width}\verb|][|\emph{pos}\verb|]{|\emph{text}\verb|}|
}

\noindent\fbox{
\verb|\framebox[|\emph{width}\verb|][|\emph{pos}\verb|]{|\emph{text}\verb|}|
}

\smallbreak

\noindent
\emph{width} defines the width of the resulting box as seen from the outside.
The \emph{pos} parameter takes a one letter value: \textbf{c}enter,
flush\textbf{l}eft, or flush\textbf{r}ight. \textbf{s}pread is not really working
in HTML.


The command \verb|framebox| works exactly the same as \verb|makebox|, but it
draws a box around the text.

The following example shows you some things you could do with
the \verb|makebox| and \verb|framebox| commands.

\begin{quote}
    \fbox{\makebox[10cm][c]{\textbf{c} e \textbf{n} t r a l}}\par

    \framebox{Guess I'm framed now!} \par
    \framebox[2cm][r]{Bummer, I am too wide} \par
    \framebox[1cm][l]{never mind, so am I}
    Can you read this?
\end{quote}


\bigbreak

\noindent\fbox{
\verb|\parbox[|\emph{pos}\verb|][|\emph{height}\verb|][|\emph{inner-pos}\verb|]{|\emph{width}\verb|}{|\emph{text}\verb|}|
}

\smallbreak

\noindent
The \verb|\parbox| command produces a box the contents of which are created in paragraph mode. However, only small
pieces of text should be used, paragraph-making environments shouldn't be used inside a \verb|\parbox| argument. For
larger pieces of text, including ones containing a paragraph-making environment, you should use a \verb|minipage|
environment.

By default LaTeX will position vertically a parbox so its center lines up with the center of the surrounding text line.
When the optional position argument is present and equal either to ‘\verb|t|’ or ‘\verb|b|’, this allows you respectively
to align either the top or bottom line in the parbox with the baseline of the surrounding text. You may also specify
‘\verb|m|’ for position to get the default behaviour.

The optional \emph{height} argument overrides the natural height of the box.

The \emph{inner-pos} argument controls the placement of the text inside the box, as follows; if it is not specified, \emph{pos} is used.

\verb|t| text is placed at the top of the box.

\verb|c| text is centered in the box.

\verb|b| text is placed at the bottom of the box.

\verb|s| is not supported in HTML

\smallbreak\noindent
The following examples demonstrate simple positioning:

\begin{itemize}
\item simple alignment:

Some text
\fbox{\parbox{2cm}{parbox default alignment, parbox test text}}
some text
\fbox{\parbox[t]{2cm}{parbox top alignment, text parbox test text}}
some text
\fbox{\parbox[b]{2cm}{parbox bottom alignment, text parbox test text}}
some text.


\item alignment with a given height:

Some text
\fbox{\parbox[c][3cm]{2cm}{parbox default alignment, parbox test text}}
some text
\fbox{\parbox[t][3cm]{2cm}{parbox top alignment, text parbox test text}}
some text
\fbox{\parbox[b][3cm]{2cm}{parbox bottom alignment, text parbox test text}}
some text.
% BUG: empty line here causes parse error!
\end{itemize}

\noindent
The following examples demonstrate all explicit \emph{pos}/\emph{inner-pos} combinations:
\begin{itemize}
\item center alignment:

\noindent
Some text
\fbox{\parbox[c][3cm][t]{2cm}{parbox default alignment, parbox test text}}
some text
\fbox{\parbox[c][3cm][c]{2cm}{parbox top alignment, text parbox test text}}
some text
\fbox{\parbox[c][3cm][b]{2cm}{parbox bottom alignment, text parbox test text}}
some text.


\item top alignment:

\noindent
Some text
\fbox{\parbox[t][3cm][t]{2cm}{parbox default alignment, parbox test text}}
some text
\fbox{\parbox[t][3cm][c]{2cm}{parbox top alignment, text parbox test text}}
some text
\fbox{\parbox[t][3cm][b]{2cm}{parbox bottom alignment, text parbox test text}}
some text.

\item bottom alignment:

\noindent
Some text
\fbox{\parbox[b][3cm][t]{2cm}{parbox default alignment, parbox test text}}
some text
\fbox{\parbox[b][3cm][c]{2cm}{parbox top alignment, text parbox test text}}
some text
\fbox{\parbox[b][3cm][b]{2cm}{parbox bottom alignment, text parbox test text}}
some text.


\item top/bottom in one line:

\noindent
Some text
\fbox{\parbox[b][3cm][t]{2cm}{parbox default alignment, parbox test text}}
some text
\fbox{\parbox[t][3cm][c]{2cm}{parbox top alignment, text parbox test text}}
some text.
\end{itemize}



\subsection{Low-level box-interface}

\LaTeX.js supports the following low-level \TeX\ commands:
%\noindent\fbox{
    \verb|\llap{|\emph{text}\verb|}|%}
,
%\noindent\fbox{
    \verb|\rlap{|\emph{text}\verb|}|%}
, and
    \verb|\smash{|\emph{text}\verb|}|, as well as
    \verb|\hphantom{|\emph{text}\verb|}|,
    \verb|\vphantom{|\emph{text}\verb|}|, and
    \verb|\phantom{|\emph{text}\verb|}|.

A phantom looks like this: \phantom{phantom} yes, now the phantom is gone.

\noindent
\llap{\verb|\llap |}could be used to put something in the margin. However, there are better alternatives for that.
See \verb|\marginpar|. \marginpar{{\huge test} in margin here}


\section{Spacing}

The following horizontal spaces are supported:
\\[8pt]
Negative thin space: |\negthinspace| \\
No space (natural): || \\
Thin space: |\,| or |\thinspace| \\
Normal space: | | \\
Normal space: |\ | \\
Non-break space: |~| \\
en-space: |\enspace| \\
em-space: |\quad| \\
2x em-space: |\qquad|\\
3cm horizontal space: |\hspace{3cm}| \\



\section{Environments}

\subsection{Lists: Itemize, Enumerate, and Description}

The \texttt{itemize} environment is suitable for simple lists, the \texttt{enumerate} environment for
enumerated lists, and the \texttt{description} environment for descriptions.

\begin{enumerate}
    \item You can nest the list environments to your taste:
        \begin{itemize}
            \item But it might start to look silly.
            \item[-] With a dash.
        \end{itemize}
    \item Therefore remember: \label{remember}
        \begin{description}
            \item[Stupid] things will not become smart because they are in a list.
            \item[Smart] things, though, can be presented beautifully in a list.
        \end{description}
    \item[important] Technical note: Viewing this in Chrome, however, will show too much vertical space
        at the end of a nested environment (see above). On top of that, margin collapsing for inline-block
        boxes is not allowed. Maybe using \texttt{dl} elements is too complicated for this and a simple nested
        \texttt{div} should be used instead.
\end{enumerate}
%
Lists can be deeply nested:
%
\begin{itemize}
    \item list text, level one
    \begin{itemize}
        \item list text, level two
        \begin{itemize}
            \item list text, level three

            And a new paragraph can be started, too.
            \begin{itemize}
                \item list text, level four

                And a new paragraph can be started, too.
                This is the maximum level.

                \item list text, level four
            \end{itemize}

            \item list text, level three
        \end{itemize}
        \item list text, level two
    \end{itemize}
    \item list text, level one
    \item list text, level one
\end{itemize}


\subsection{Flushleft, Flushright, and Center}

The \texttt{flushleft} environment:
%
\begin{flushleft}
This text is\\ left-aligned.
\LaTeX{} is not trying to make
each line the same length.
\end{flushleft}
%
The \texttt{flushright} environment:
%
\begin{flushright}
This text is right-\\aligned.
\LaTeX{} is not trying to make
each line the same length.
\end{flushright}
%
And the \texttt{center} environment:
%
\begin{center}
At the centre\\of the earth
\end{center}



\subsection{Quote, Quotation, and Verse}

The \texttt{quote} environment is useful for quotes, important phrases and examples.
A typographical rule of thumb for the line length is:
\begin{quote}
On average, no line should be longer than 66 characters.
\end{quote}

There are two similar environments: the \texttt{quotation} and the \texttt{verse} environments.
The \texttt{quotation} environment is useful for longer quotes going over several paragraphs,
because it indents the first line of each paragraph.

The \texttt{verse} environment is useful for poems where the line breaks are important.
The lines are separated by issuing a \texttt{\textbackslash\textbackslash} at the end of a line
and an empty line after each verse.

\begin{verse}
Humpty Dumpty sat on a wall:\\
Humpty Dumpty had a great fall.\\
All the King’s horses and all the King’s men\\
Couldn’t put Humpty together again.

{\raggedleft ---J.W. Elliott\par}
\end{verse}


\subsection{Picture}

\frame{\setlength{\unitlength}{20.4mm}
\begin{picture}(3,2.1)(-1.2,-0.05)
    \put(0,1){\vector(1,0){1}}
    \put(0,1){\circle{2}}
    \thicklines
    \put(0,0){\line(1,0){1}}
    \put(0,0.01){xxxxxxxxxxx}
    \put(0,0.1){XXXX}
\end{picture}}
%
\frame{\setlength{\unitlength}{1mm}
\begin{picture}(60, 50)
    \put(20,30){\circle{1}}
    \put(20,30){\circle{2}}
    \put(20,30){\circle{4}}
    \put(20,30){\circle{8}}
    \put(20,30){\circle{16}}
    \put(20,30){\circle{32}}
    \put(40,30){\circle{1}}
    \put(40,30){\circle{2}}
    \put(40,30){\circle{3}}
    \put(40,30){\circle{4}}
    \put(40,30){\circle{5}}
    \put(40,30){\circle{6}}
    \put(40,30){\circle{7}}
    \put(40,30){\circle{8}}
    \put(40,30){\circle{9}}
    \put(40,30){\circle{10}}
    \put(40,30){\circle{11}}
    \put(40,30){\circle{12}}
    \put(40,30){\circle{13}}
    \put(40,30){\circle{14}}
    \put(15,10){\circle*{1}}
    \put(20,10){\circle*{2}}
    \put(25,10){\circle*{3}}
    \put(30,10){\circle*{4}}
    \put(35,10){\circle*{5}}
\end{picture}}

\frame{\setlength{\unitlength}{0.75mm}
\begin{picture}(60,40)
    \put(30,20){\vector(1,0){30}}
    \put(30,20){\vector(4,1){20}}
    \put(30,20){\vector(3,1){25}}
    \put(30,20){\vector(2,1){30}}
    \put(30,20){\vector(1,2){10}}
    \thicklines
    \put(30,20){\vector(-4,1){30}}
    \put(30,20){\vector(-1,4){5}}
    \thinlines
    \put(30,20){\vector(-1,-1){5}}
    \put(30,20){\vector(-1,-4){5}}
\end{picture}}
%
\setlength{\unitlength}{5cm}
\begin{picture}(1,1)
    \put(0,0){\line(0,1){1}}
    \put(0,0){\line(1,0){1}}
    \put(0,0){\line(1,1){1}}
    \put(0,0){\line(1,2){.5}}
    \put(0,0){\line(1,3){.3333}}
    \put(0,0){\line(1,4){.25}}
    \put(0,0){\line(1,5){.2}}
    \put(0,0){\line(1,6){.1667}}
    \put(0,0){\line(2,1){1}}
    \put(0,0){\line(2,3){.6667}}
    \put(0,0){\line(2,5){.4}}
    \put(0,0){\line(3,1){1}}
    \put(0,0){\line(3,2){1}}
    \put(0,0){\line(3,4){.75}}
    \put(0,0){\line(3,5){.6}}
    \put(0,0){\line(4,1){1}}
    \put(0,0){\line(4,3){1}}
    \put(0,0){\line(4,5){.8}}
    \put(0,0){\line(5,1){1}}
    \put(0,0){\line(5,2){1}}
    \put(0,0){\line(5,3){1}}
    \put(0,0){\line(5,4){1}}
    \put(0,0){\line(5,6){.8333}}
    \put(0,0){\line(6,1){1}}
    \put(0,0){\line(6,5){1}}
\end{picture}


\frame{
    \setlength{\unitlength}{1cm}
    \begin{picture}(6,5)
    \thicklines
    \put(1,0.5){\line(2,1){3}}
    \put(4,2){\line(-2,1){2}}
    \put(2,3){\line(-2,-5){1}}
    \put(0.7,0.3){$A$}
    \put(4.05,1.9){$B$}
    \put(1.7,2.9){$C$}
    \put(3.1,2.5){$a$}
    \put(1.3,1.7){$b$}
    \put(2.5,1){$c$}
    \put(0.3,4){$F=\sqrt{s(s-a)(s-b)(s-c)}$}
    \put(3.5,0.4){$\displaystyle s:=\frac{a+b+c}{2}$}
    \end{picture}
}

\setlength{\unitlength}{2mm}
\begin{picture}(30,20)
    \linethickness{0.075mm}
    \multiput(0,0)(1,0){26}{\line(0,1){20}}
    \multiput(0,0)(0,1){21}{\line(1,0){25}}
    \linethickness{0.15mm}
    \multiput(0,0)(5,0){6}{\line(0,1){20}}
    \multiput(0,0)(0,5){5}{\line(1,0){25}}
    \linethickness{0.3mm}
    \multiput(5,0)(10,0){2}{\line(0,1){20}}
    \multiput(0,5)(0,10){2}{\line(1,0){25}}
\end{picture}
%
\setlength{\unitlength}{0.7cm}
\begin{picture}(6,4)
    \linethickness{0.075mm}
    \multiput(0,0)(1,0){7}{\line(0,1){4}}
    \multiput(0,0)(0,1){5}{\line(1,0){6}}
    \thicklines
    \put(2,3){\oval(3,1.8)}
    \thinlines
    \put(3,2){\oval(3,1.8)}
    \thicklines
    \put(2,1){\oval(3,1.8)[tl]}
    \put(4,1){\oval(3,1.8)[b]}
    \put(4,3){\oval(3,1.8)[r]}
    \put(3,1.5){\oval(1.8,0.4)}
\end{picture}

\setlength{\unitlength}{0.8cm}
\begin{picture}(6,4)
    \linethickness{0.075mm}
    \multiput(0,0)(1,0){7}{\line(0,1){4}}
    \multiput(0,0)(0,1){5}{\line(1,0){6}}
    \thicklines
    \put(0.5,0.5){\line(1,5){0.5}}
    \put(1,3){\line(4,1){2}}
    \qbezier(0.5,0.5)(1,3)(3,3.5)
    \thinlines
    \put(2.5,2){\line(2,-1){3}}
    \put(5.5,0.5){\line(-1,5){0.5}}
    \linethickness{1mm}
    \qbezier(2.5,2)(5.5,0.5)(5,3)
    \thinlines
    \qbezier(4,2)(4,3)(3,3)
    \qbezier(3,3)(2,3)(2,2)
    \qbezier(2,2)(2,1)(3,1)
    \qbezier(3,1)(4,1)(4,2)
\end{picture}



\section{Labels and References}

Labels can be attached to parts, chapters, sections, items of enumerations, footnotes, tables and figures.
For instance: item~\ref{remember} was important, and regarding fonts, read Section~\ref{sec:advice}. And
below, we can reference item~\ref{key-1} and \ref{key-2}.

\begin{enumerate}
    \item list text, level one
    \begin{enumerate}
        \item list text, level two
        \begin{enumerate}
            \item list text, level three

            And a new paragraph can be started, too.
            \begin{enumerate}
                \item list text, level four

                And a new paragraph can be started, too.
                This is the maximum level.

                \item list text, level four \label{key-1}
            \end{enumerate}

            \item list text, level three
        \end{enumerate}
        \item\label{key-2} list text, level two
    \end{enumerate}
    \item list text, level one
    \item list text, level one
\end{enumerate}


\section{Mathematical Formulae}

Math is typeset using KaTeX. Inline math:
$
f(x) = \int_{-\infty}^\infty \hat f(\xi)\,e^{2 \pi i \xi x} \, d\xi
$
as well as display math is supported:
$$
f(n) = \begin{cases} \frac{n}{2}, & \text{if } n\text{ is even} \\ 3n+1, & \text{if } n\text{ is odd} \end{cases}
$$


\section{Groups}


Today is \today.

Actually, what about { some groups? } They\,are\ \ \ \ \ nice.


\section{Symbols}
\label{sec:symbols}

\noindent
lowercase greek letters:
\textalpha \textbeta \textgamma \textdelta \textepsilon \textzeta \texteta \texttheta \textiota \textkappa
\textlambda \textmu \textnu \textxi \textomikron \textpi \textrho \textsigma \texttau \textupsilon \textphi \textchi
\textpsi \textomega

\noindent
uppercase greek letters:
\textAlpha \textBeta \textGamma \textDelta \textEpsilon \textZeta \textEta \textTheta \textIota \textKappa
\textLambda \textMu \textNu \textXi \textOmikron \textPi \textRho \textSigma \textTau \textUpsilon \textPhi \textChi
\textPsi \textOmega

\noindent
currencies:
\texteuro \textcent \textsterling \pounds \textbaht \textcolonmonetary \textcurrency \textdong \textflorin \textlira
\textnaira \textpeso \textwon \textyen

\noindent
old-style numerals:
\textzerooldstyle \textoneoldstyle \texttwooldstyle \textthreeoldstyle \textfouroldstyle \textfiveoldstyle
\textsixoldstyle \textsevenoldstyle \texteightoldstyle \textnineoldstyle

\noindent
math:
\textperthousand \perthousand \textpertenthousand \textonehalf \textthreequarters \textonequarter
\textfractionsolidus \textdiv \texttimes \textminus \textpm \textsurd \textlnot \textasteriskcentered
\textonesuperior \texttwosuperior \textthreesuperior

\noindent
arrows:
\textleftarrow \textuparrow \textrightarrow \textdownarrow

\noindent
misc:
\checkmark \textreferencemark \textordfeminine \textordmasculine \textmarried \textdivorced \textbar \textbardbl
\textbrokenbar \textbigcircle \textcopyright \copyright \textcircledP \textregistered \textservicemark
\texttrademark \textnumero \textrecipe \textestimated \textmusicalnote \textdiscount

\noindent
non-ASCII:
\AE \ae \IJ \ij \OE \oe \TH \th \SS \ss \DH \dh \O \o \DJ \dj \L \l \i \j \NG \ng


\section{Fonts}

Usually, \LaTeX.js chooses the right font---just like \LaTeX.  In some cases,
one might like to change fonts and sizes by hand. To do this, use the standard
commands. The actual size of each font is a design issue and depends
on the document class (in this case on the CSS file).

{\small The small and
    \textbf{bold} Romans ruled}
    {\Large all of great big
    \textit{Italy}.}

\textit{You can also
    \emph{emphasize} text if
    it is set in italics,}
    \textsf{in a
    \emph{sans-serif} font,}
    \texttt{or in
    \emph{typewriter} style.}

The environment form of the font commands is available, too:

\begin{center}
\begin{itshape}
This whole paragraph is emphasized, for instance.
\end{itshape}
\end{center}


\subsection{An advice}
\label{sec:advice}
\begin{center}
    \underline{\textbf{Remember\Huge!}} \textit{The}
    \textsf{M\textbf{\LARGE O}\texttt{R}\textsl{E}} fonts \Huge you
    \tiny use \footnotesize \textbf{in} a \small \texttt{document},
    \large \textit{the} \normalsize more \textsc{readable} and
    \textsl{\textsf{beautiful} it bec\large o\Large m\LARGE e\huge s}.
\end{center}


\appendix

\section{Source}

The source of \LaTeX.js is here on GitHub: \url{https://github.com/michael-brade/LaTeX.js}

\end{document}
