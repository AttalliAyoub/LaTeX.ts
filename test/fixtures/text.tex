** simple text in paragraphs
.
This is just some simple text.

This is just some simple text.

This is just some simple text.
.
<p>This is just some simple text.</p>
<p>This is just some simple text.</p>
<p>This is just some simple text.</p>
.


** paragraphs and newlines
.
first line\\second line\newline third line\par and then a new paragraph as well
.
<p>first line<br>second line<br>third line</p>
<p>and then a new paragraph as well</p>
.


** paragraphs and indentation
.
\noindent

\noindent

This is a new paragraph.

\noindent
This is a new paragraph.
.
<p>This is a new paragraph.</p>
<p class="noindent">This is a new paragraph.</p>
.


** UTF-8 text, punctuation, TeX symbols
.
A para\-graph with “quotes,” German »quotes«, umlauts (äöüÄÖÜ), ß, thin spaces 80\,000. Numbers?! Money \$10, \$5, and 5€.

.:;,?!'´`()[]-/*@+=

H\^otel, na\"\i ve, \'el\`eve, sm\o rrebr\o d, !´Se\~norita!, Sch\"onbrunner Schlo\ss{} Stra\ss e
.
<p>A para­graph with “quotes,” German »quotes«, umlauts (äöüÄÖÜ), ß, thin spaces 80 000. Numbers?! Money $10, $5, and 5€.</p>
<p>.:;,?!’´‘()[]‐/*@+=</p>
<p>Hôtel, naı̈ve, élève, smørrebrød, ¡Señorita!, Schönbrunner Schloß​ Straße</p>
.


** special characters
.
\# \$ \$ \^{} \& \_ \{ \} \~{} \textbackslash{} \% shelf\-ful
.
<p># $ $ ^​ &amp; _ { } ~​ \​ % shelf­ful</p>
.


** dashes, dots (no math)
.
daughter-in-law, pages 13--67, yes---or no?
.
<p>daughter‐in‐law, pages 13–67, yes—or no?</p>
.


** ligatures, and ligature prevention
.
ff ffi ffl fi fl << >> !´ ?´

Not shelfful but shelf\mbox{}ful, or better, shelf\/ful.
.
<p>ff ffi ffl fi fl « » ¡ ¿</p>
<p>Not shelfful but shelf<span class="hbox"><span></span></span>ful, or better, shelf‌ful.</p>
.


** TeX and LaTeX logos
.
\TeX \LaTeX
.
<p>
<span class="tex">T<span class="e">e</span>X</span>
<span class="latex">L<span class="a">a</span>T<span class="e">e</span>X</span>
</p>
.


s** alignment
.
This is a horrible test.
\centering
In this paragraph we change the alignment to centering.
{\raggedright But it actually becomes raggedright,

even with a group, and only after the par will it be centered.}
Until the group ends.

And we are now still centered.
.
<p class="raggedright">This is a horrible test. In this paragraph we change the alignment to centering. But it actually becomes raggedright,</p>
<p class="centering">even with a group, and only after the par will it be centered.​ Until the group ends.</p>
<p class="centering">And we are now still centered.</p>
.
