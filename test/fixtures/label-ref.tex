** forward- and back-referencing a label
.
\newcounter{c}

Here is some text with two forward references: \ref{test} and \ref{test}.
\refstepcounter{c}
Now there is more.

This \label{test} is counter c:~\ref{test}. And another~\ref{test}.
.
<p>Here is some text with two forward references: <a href="#c-1">1</a> and <a href="#c-1">1</a>. <span id="c-1"></span> Now there is more.</p>
<p>This  is counter c:&nbsp;<a href="#c-1">1</a>. And another&nbsp;<a href="#c-1">1</a>.</p>
.


** forward-referencing a section
.
See Section~\ref{sec:test}.

\section{Section Name}
\label{sec:test}
.
<p>See Section&nbsp;<a href="#sec-1">1</a>.</p>
<h2 id="sec-1">1 Section Name</h2>
.


** back-referencing a section
.
\section{Section Name}

This \label{sec:test} is Section~\ref{sec:test}.
.
<h2 id="sec-1">1 Section Name</h2>
<p>This is Section&nbsp;<a href="#sec-1">1</a>.</p>
.


** forward- and back-referencing a section in a chapter
.
\chapter{Labels}

See Section~\ref{sec:test}.

\section{Section Name}
\label{sec:test}
This is Section~\ref{sec:test}.
.
<h1 id="sec-1">1 Labels</h1>
<p>See Section&nbsp;<a href="#sec-2">1.1</a>.</p>
<h2 id="sec-2">1.1 Section Name</h2>
<p>This is Section&nbsp;<a href="#sec-2">1.1</a>.</p>
.


** labelling an \\item
.
The most widely-used format is item number~\ref{popular}.
\begin{enumerate}
    \item Plain \TeX
    \item \label{popular} \LaTeX
    \item Con\TeX t
\end{enumerate}
.
.



!** referencing subitems
.
.
.


!** labelling a figure
.
\begin{figure}
    ...
    \caption{caption text}
    \label{fig:test}
\end{figure}
See Figure~\ref{fig:test}.
.
.


!** labelling a table
.
\begin{table}
    ...
    \caption{caption text}
    \label{tab:test}
\end{table}
See Table~\ref{tab:test}.
.
.
